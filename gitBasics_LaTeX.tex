%!TEX TS-program = pdflatex
\documentclass[12pt,ngerman]{beamer}

\usepackage[utf8]{inputenc}
\usepackage[T1]{fontenc}
\usepackage{booktabs}
\usepackage{babel}
\usepackage{graphicx}
\usepackage{csquotes}
\usepackage{xcolor}
\usepackage{nicefrac}
\usepackage{siunitx}


\IfFileExists{plex-sans.sty}{%
\usepackage[sfdefault]{plex-sans}%
}{
% no plex sans
}

\usetheme[progressbar=frametitle]{metropolis}           % Use metropolis theme
 

 
\makeatletter
\setlength{\metropolis@titleseparator@linewidth}{1pt}
\setlength{\metropolis@progressonsectionpage@linewidth}{1pt}
\setlength{\metropolis@progressinheadfoot@linewidth}{1pt}
\makeatother

\title{Introduction to \texttt{git}}
\author{Dr. Uwe Ziegenhagen}
\institute{\url{www.uweziegenhagen.de}}

\begin{document}

\begin{frame}

\maketitle

\end{frame}

\begin{frame}

\tableofcontents

\end{frame}

\section{Backups and Version Control}

\begin{frame}[allowframebreaks]
\frametitle{Why you need Backups!}
\framesubtitle{~}

\vspace*{1em}
\begin{quote}
Meiner Freundin wurde letzte Woche Ihre Tasche gestohlen und nun möchte ich für sie einen dringenden Aufruf starten, denn dabei war auch ihr Laptop mit der gesamten Masterthesis!!!! die in 3 Wochen abgegeben werden muss. 
\end{quote}

\vspace*{1em}
\begin{quote}
Notfall!!!! Eine Freundin von mir muss morgen ihre Hausarbeit abgeben und nun ist ihr Macbook kaputt. Festplatte ist hin. Wer von euch an ihr für ne kleine Mark helfen? Die Hausarbeit muss da nur runter aufn stick oder so. Es ist dringend!!!! 
\end{quote}

\vspace*{1em}
\begin{quote}
Mir ist meine externe Festplatte runtergefallen...  Sie geht nach wie vor an (hat einen Schalter dafür an der Rückseite), man hört auch noch das klassische \enquote{USB-Anschluss-Geräusch} am Laptop, aber ich kann sie nicht mehr sehen geschweige denn auf Dateien zugreifen.
\end{quote}

\end{frame}

\begin{frame}[fragile,allowframebreaks]
\frametitle{Why you need Version Control!}
\framesubtitle{~}

\vspace*{2em}
\begin{verbatim}
Thesis_final.docx
Thesis_final-Version_20180410.docx
Thesis_final-Version_20180411.docx
Thesis_final-Version_final_12.04.2018.docx
Thesis_final-Version_final_12.04.2018_remarks.docx
\end{verbatim}

\vspace*{3em}

\begin{quote}
\enquote{Hier sind meine Änderungen für das Paper und die Änderungen der anderen vier Autoren, jeweils in einer Datei. Ich weiß, Du hast längst eine neuere Version, aber Du bekommst das schon hin. Brauch ich bis Sonntagabend!}

\end{quote}

\vspace*{3em}

\begin{quote}
\enquote{Nimm mal doch die Änderungen raus, die ich Dir vor drei Monaten genannt hatte. Lass aber die drin, die ich Dir vor zwei Monaten geschickt hatte.}
\end{quote}

\end{frame}

\begin{frame}
\frametitle{Conclusion}
\framesubtitle{~}

\begin{itemize}
\item Backups are essential!
\item Backups on the same machine are very dangerous!
\item Version control can act as backup but does much more:

\begin{itemize}
	\item Integrate code from various authors
	\item Go back to when the code was still working
	\item 
\end{itemize}

\end{itemize}
\end{frame}


\section{Central versus Distributed VC}

\begin{frame}
\frametitle{}
\framesubtitle{~}

\begin{itemize}
\item 
\item 
\item 
\item 
\item 
\item 
\end{itemize}
\end{frame}

\section{Introduction to git}

\begin{frame}
\frametitle{}
\framesubtitle{~}

\begin{itemize}
\item 
\item 
\item 
\item 
\item 
\item 
\end{itemize}
\end{frame}


\end{document}

https://www.git-tower.com/learn/git/ebook/en/desktop-gui/basics/why-use-version-control